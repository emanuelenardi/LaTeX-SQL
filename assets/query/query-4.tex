\begin{lstlisting}
SELECT	Impiegato.Nome, Impiegato.Cognome, Dipartimento.Citta
FROM	Impiegato, Dipartimento
WHERE	Impiegato.Dipart = Dipartimento.Nome
\end{lstlisting}

La query produce la seguente tabella:

\begin{center}
	\begin{tabular}{@{} l l l @{}}
		\toprule
			Impiegato.Nome	& Impiegato.Cognome & Dipartimento.Citta \\
		\midrule
			Mario			& Rossi				& Milano \\
			\dots			& \dots				& \dots \\
			Marco			& Franco			& Torino \\
		\bottomrule
	\end{tabular}
\end{center}

\'E necessario specificare il nome della tabella quando le tabelle presenti nella clausola \texttt{FROM} posseggono più attributi con lo stesso nome. Qualora non vi sia possibilità di ambiguità è possibile specificare l'attributo senza dichiarare la tabella di appartenenza.
