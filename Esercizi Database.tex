% ---------------------------------------------------------------------------- %
% ---------------------------------------------------------------------------- %
% ---------------------------------------------------------------------------- %

% ---------------------------------------------------------------------------- %
\author{Emanuele Nardi}
\title{Esercizi Database}

\newcommand{\pdfTitle}{UNITN -- Esercizi Database}
\newcommand{\tags}{SQL}

% ---------------------------------------------------------------------------- %

\newcommand{\logo}{\includegraphics[width=0.6\textwidth, keepaspectratio]{logo_unitn_black_center}}
\newcommand{\DISI}{Dipartimento di Ingegneria e Scienza dell'Informazione}
\newcommand{\faculty}{Corso di Laurea in\\ Informatica}

% ---------------------------------------------------------------------------- %

\newcommand{\what}{Esercizi di}
\newcommand{\subject}{Database}
\newcommand{\prof}{Prof.re Yannis Velegrakis}

\newcommand{\authorName}{Emanuele Nardi}
\newcommand{\reviserName}{}
\newcommand{\academicYear}{2017/2018}

% ---------------------------------------------------------------------------- %

% \documentclass[a4paper, 10pt]{article}

\usepackage[utf8]{inputenc}
\usepackage[T1]{fontenc}
\usepackage[italian]{babel}

\usepackage{
	array,
	beramono,
	booktabs,
	color,
	float,
	geometry,
	listings,
	listingsutf8,
	tabularx,
	tikz,
	xcolor,
}

\documentclass[a4paper, 10pt]{article}

\usepackage[utf8]{inputenc}
\usepackage[T1]{fontenc}
\usepackage[italian]{babel}

\usepackage{
	array,
	beramono,
	booktabs,
	color,
	float,
	geometry,
	listings,
	listingsutf8,
	tabularx,
	tikz,
	xcolor,
}

\input{mystyle}

\geometry {
	a4paper,
	top = 3cm,
	bottom = 3cm,
	left = 2cm,
	right = 2cm,
}

\lstdefinestyle{SQL}{
	language = {SQL},
	basicstyle = \footnotesize\ttfamily,
	commentstyle = \color{ashgrey},
	keywordstyle = \color{black}\bfseries,
	moredelim=**[is][\btHL]{^}{^},
	moredelim=**[is][{\btHL[fill=green!30]}]{@}{@},
	breakatwhitespace = false,
	breaklines = true,
	escapeinside = {(*@}{@*)},
	morekeywords = {},
	tabsize = 4,
	frame = single,
	captionpos = b,
	aboveskip = 3mm,
	belowskip = 3mm,
}

\lstset{
	inputencoding = utf8/latin1,
	extendedchars = false,
	language = SQL,
	style = SQL
}

\definecolor{ashgrey}{rgb}{0.7, 0.75, 0.71}
\renewcommand\labelitemi{--}

\def\firstcircle{(210:1.75cm) circle (2cm)}
\def\secondcircle{(330:1.75cm) circle (2cm)}
\renewcommand{\arraystretch}{1.2}


% ---------------------------------------------------------------------------- %
% ---------------------------------------------------------------------------- %

\begin{document}

% ---------------------------------------------------------------------------- %

	\input{initial_page}
	\afterpage{\blankpage}
	\clearpage

% ---------------------------------------------------------------------------- %
% ---------------------------------------------------------------------------- %

	\section*{Introduzione}

\emph{Qusto è un estratto dal capitolo 4.3 "Interrogazioni SQL" del libro "Basi di Dati" edito da McGrawHill IV ed. versione italiana.}

Lo scopo principale di questi appunti è quello di svolgere esercizio in preparazione all'esame di Database tenuto all'Università degli Studi di Trento.

Queste note non sono complete, e la loro lettura non permette, da sola, di superare l’esame. La versione più recente di queste note si trova all'indirizzo:

\begin{center}
	\url{https://github.com/emanuelenardi/latex-sql}
\end{center}

\subsection*{Materiale}

Per tutto il resto consulta la cartella Google Drive del corso di Informatica %
\href{https://bit.ly/drive-folder}{\ExternalLink}.

\subsection*{Segnalazione di errori}

Se hai trovato un errore ti prego di inviarmi un'e-mail \href{mailto:emanuele.nardi@studenti.unitn.it}{\ExternalLink} allegando un esempio che possa riprodurre l'errore.

% \subsection*{Ringraziamenti}

	\input{./assets/other/self-intro}
	\afterpage{\blankpage}
	\clearpage

% ---------------------------------------------------------------------------- %
% ---------------------------------------------------------------------------- %

\begin{center}
	\section*{Tabelle}
\end{center}

\begin{table}[H]
	\centering
	\begin{tabular}{@{} l l l l l l @{}}
		\toprule
			Nome 		& Cognome 		& Dipart 			& Ufficio	& Stipendio & Citta \\
		\midrule
			Mario		& Rossi			& Amministrazione	& 10		& 45		& Milano \\
			Carlo		& Bianchi		& Produzione		& 20		& 36		& Torino \\
			Giovanni	& Verdi			& Amministrazione	& 20		& 40		& Roma \\
			Franco		& Neri			& Distribuzione		& 16		& 45		& Napoli \\
			Carlo		& Rossi			& Direzione			& 14		& 80		& Milano \\
			Lorenzo		& Gialli		& Direzione			& 7			& 73		& Genova \\
			Paola		& Rosati		& Amministrazione	& 75		& 40		& Venezia \\
			Marco		& Franco		& Produzione		& 20		& 46		& Roma \\
		\bottomrule
	\end{tabular}
	\caption{IMPIEGATO}
\end{table}


\begin{table}[H]
	\centering
	\begin{tabular}{@{}llll@{}}
		\toprule
			CodFiscale 	& Nome 		& Cognome	& Città \\
		\midrule
			x			& Mario		& Rossi		& Verona \\
			x			& Carlo		& Bianchi	& Roma \\
			x			& Giovanni	& Rossi		& Verona \\
			x			& Pietro	& Rossi		& Milano \\
		\bottomrule
	\end{tabular}
	\caption{PERSONA}
\end{table}

\begin{table}[H]
	\centering
	\begin{tabular}{@{} l l l @{}}
		\toprule
			Nome 				& Indirizzo				& Città \\
		\midrule
			Amministrazione		& Via Tito Livio, 27	& Milano \\
			Produzione			& P.le Lavater, 3		& Torino \\
			Distribuzione		& Via Sagra, 9			& Roma \\
			Direzione			& Via Tito Livio, 27	& Milano \\
			Ricerca				& Via Venosa, 6			& Milano \\
		\bottomrule
	\end{tabular}
	\caption{DIPARTIMENTO}
\end{table}


% ---------------------------------------------------------------------------- %
% ---------------------------------------------------------------------------- %

\newpage
\section*{Interrogazioni semplici}

\begin{enumerate}
	\item Estrarre lo stipendio degli impiegati di cognome "Rossi".
	\begin{lstlisting}
SELECT Stipendio AS Salario
FROM   Impiegato
WHERE  Cognome = 'Rossi'
\end{lstlisting}

La query produce la seguente tabella:

\begin{center}
	\begin{tabular}{@{} l @{}}
		\toprule
			Salario \\
		\midrule
			45 \\
			80 \\
		\bottomrule
	\end{tabular}
\end{center}


	\item Estrarre tutte le informazioni relative agli impiegati di cognome "Rossi".
	\begin{lstlisting}
SELECT *
FROM   Impiegato
WHERE  Cognome = 'Rossi'
\end{lstlisting}

La query produce la seguente tabella:

\begin{center}
	\begin{tabular}{@{}llllll@{}}
		\toprule
			Nome 		& Cognome 		& Dipart 			& Ufficio	& Stipendio & Città \\
		\midrule
			Mario		& Rossi			& Amministrazione	& 10		& 45		& Milano \\
			Carlo		& Rossi			& Direzione			& 14		& 80		& Milano \\
		\bottomrule
	\end{tabular}
\end{center}


	\item Estrarre lo stipendio mensile dell'impiegato che ha cognome "Bianchi".
	\begin{lstlisting}
SELECT Stipendio/12 AS StipendioMensile
FROM   Impiegato
WHERE  Cognome = 'Bianchi'
\end{lstlisting}

La query produce la seguente tabella:

\begin{center}
	\begin{tabular}{@{} l @{}}
		\toprule
			StipendioMensile \\
		\midrule
			3,00 \\
		\bottomrule
	\end{tabular}
\end{center}


	\item Estrarre il nome degli impiegati e le citta in cui lavorano.
	\begin{lstlisting}
SELECT	Impiegato.Nome, Impiegato.Cognome, Dipartimento.Citta
FROM	Impiegato, Dipartimento
WHERE	Impiegato.Dipart = Dipartimento.Nome
\end{lstlisting}

La query produce la seguente tabella:

\begin{center}
	\begin{tabular}{@{} l l l @{}}
		\toprule
			Impiegato.Nome	& Impiegato.Cognome & Dipartimento.Citta \\
		\midrule
			Mario			& Rossi				& Milano \\
			\dots			& \dots				& \dots \\
			Marco			& Franco			& Torino \\
		\bottomrule
	\end{tabular}
\end{center}

\'E necessario specificare il nome della tabella quando le tabelle presenti nella clausola \texttt{FROM} posseggono più attributi con lo stesso nome. Qualora non vi sia possibilità di ambiguità è possibile specificare l'attributo senza dichiarare la tabella di appartenenza.


	\item L'interpretazione precedente può essere espressa facendo uso degli \emph{alias} per la tabella allo scopo di abbreviare i riferimenti a esse.
	\begin{lstlisting}
SELECT I.Nome, Cognome, D.Citta
FROM   Impiegato AS I, Dipartimento AS D
WHERE  Dipart = D.Nome -- non presentano ambiguita'
\end{lstlisting}

Utilizzando gli alias è possibile fare accesso più volte alla stessa tabella, come avviene nel calcolo relazionale. Tutte le volte che si introduce un alias per una tabella si dichiara in effetti una variabile che rappresenta le righe della tabella di cui è alias.

\medskip
Quando una tabella una sola volta in un interrogazione, non c'è differenza tra l'interpretare l'alias come uno pseudonimo o come una nuova variabile.

\medskip
Quando una tabella compare invece più volte, è necessario considerare l'alias come una nuova variabile.


	\item Estrarre il nome e il cognome degli impiegati che lavorano nell'ufficio 20 del dipartimento Amministrazione.
	\begin{lstlisting}
SELECT Nome, Cognome
FROM   Impiegato
WHERE  Ufficio = 20
  AND  Dipart = 'Amministrazione'
\end{lstlisting}

La query produce la seguente tabella:

\begin{center}
	\begin{tabular}{@{} l l @{}}
		\toprule
			Nome		& Cognome \\
		\midrule
			Giovanni	& Verdi \\
		\bottomrule
	\end{tabular}
\end{center}


	\item Estrarre i nomi e i cognomi degli impiegati che lavorano nel dipartimento Amministrazione \textbf{o} nel dipartimento Produzione.
	\begin{lstlisting}
SELECT Nome, Cognome
FROM   Impiegato
WHERE  Dipart = 'Amministrazione'
  AND  Dipart = 'Produzione'
\end{lstlisting}

La query produce la seguente tabella:

\begin{center}
	\begin{tabular}{@{} l l @{}}
		\toprule
			Nome		& Cognome \\
		\midrule
			Mario		& Rossi \\
			Carlo		& Bianchi \\
			Giovanni	& Verdi \\
			Paola		& Rosati \\
			Marco		& Franco \\
		\bottomrule
	\end{tabular}
\end{center}


	\item Estrarre i nomi propri degli impiegati di cognome "Rossi" che lavorano nei dipartimenti Amministrazione \textbf{e} Produzione.
	\begin{lstlisting}
SELECT Nome
FROM   Impiegato
WHERE  Cognome = 'Rossi'
  AND  (Dipart = 'Amministrazione' OR Dipart = 'Produzione')
\end{lstlisting}

La query produce la seguente tabella:

\begin{center}
	\begin{tabular}{@{} l @{}}
		\toprule
			Nome \\
		\midrule
			Mario \\
		\bottomrule
	\end{tabular}
\end{center}


	\item Estrarre gli impiegati che hanno un cognome che ha una "o" in seconda posizione e finisce per "i".
	\begin{lstlisting}
SELECT *
FROM   Impiegato
WHERE  Cognome LIKE '_o%i'
\end{lstlisting}

\begin{center}
	\begin{tabular}{@{} l l l l l l @{}}
		\toprule
			Nome 		& Cognome 		& Dipart 			& Ufficio	& Stipendio & Citta \\
		\midrule
			Mario		& Rossi			& Amministrazione	& 10		& 45		& Milano \\
			Carlo		& Rossi			& Direzione			& 14		& 80		& Milano \\
			Paola		& Rosati		& Amministrazione	& 75		& 40		& Venezia \\
		\bottomrule
	\end{tabular}
\end{center}


	\item Estrarre le città delle persone il cui cognome è "Rossi" presentando eventualmente più volte lo stesso valore di città.
	\begin{lstlisting}
SELECT Citta
FROM   Persona
WHERE  Cognome = 'Rossi'
\end{lstlisting}

\begin{center}
	\begin{tabular}{@{} l @{}}
		\toprule
			Citta \\
		\midrule
			Verona \\
			Verona \\
			Milano \\
		\bottomrule
	\end{tabular}
\end{center}


	\item Estrarre le città delle persone con cognome "Rossi", facendo comparire ogni città al più una volta.
	\begin{lstlisting}
SELECT DISTINCT Citta
FROM   Persona
WHERE  Cognome = 'Rossi'
\end{lstlisting}

\begin{center}
	\begin{tabular}{@{} l @{}}
		\toprule
			Citta \\
		\midrule
			Verona \\
			Milano \\
		\bottomrule
	\end{tabular}
\end{center}


% ---------------------------------------------------------------------------- %

\setcounter{enumi}{16}
	\item Estrare tutti gli impiegati che hanno lo stesso cognome (ma diverso nome) di impiegati del dipartimento Produzione.
	\begin{lstlisting}
SELECT I1.Cognome, I1.Nome
FROM   Impiegato I1, Impiegato I2
WHERE  I1.Cognome = I2.Cognome
  AND  I1.Nome <> I2.Cognome
  AND  I2.Dipart = 'Produzione'
\end{lstlisting}


% ---------------------------------------------------------------------------- %
% ---------------------------------------------------------------------------- %

\newpage
\subsection*{Operatori Aggregati}

\setcounter{enumi}{19}
	\item Estrarre il no. di impiegati del dipartimento Produzione.
	\begin{lstlisting}
SELECT count(*)
FROM   Impiegato
WHERE  Dipart = 'Produzione'
\end{lstlisting}

Il numero di impiegati corrisponderà al numero di tuple della relazione \texttt{IMPIEGATO} che possiedono "Produzione" come valore dell'attributo \texttt{Dipart}.

Operatori aggregati:

\begin{enumerate}
	\item \texttt{count}
	\item \texttt{sum}
	\item \texttt{max}
	\item \texttt{min}
	\item \texttt{avg}
\end{enumerate}


	\item Estrarre il no. di diversi valori dell'attributo Stipendio fra tutte le righe di impiegato.
	\begin{lstlisting}
SELECT count(DISTINCT Stipendio)
FROM   Impiegato
\end{lstlisting}


	\item Estrarre il no. di righe che possiedono un valore non nulla per l'attributo Nome.
	\begin{lstlisting}
SELECT count(ALL Nome)
FROM   Impiegato
\end{lstlisting}


	\item Estrarre la somma degli stipendi del dipartimento Amministrazione.
	\begin{lstlisting}
SELECT sum(Stipendio)
FROM   Impiegato
WHERE  Dipart = 'Amministrazione'
\end{lstlisting}


	\item Estrarre gli stipendi minimo, massimo e medio fra quelli di tutti gli impiegati.
	\begin{lstlisting}
SELECT min(Stipendio), max(Stipendio), avg(Stipendio)
FROM   Impiegato
\end{lstlisting}


	\item Estarre il massimo stipendio tra quelli degli impiegati che lavorano in un dipartimento con sede a Milano.
	\begin{lstlisting}
SELECT max(Stipendio)
FROM   Impiegato, Dipartimento
WHERE  Dipartimento.Citta = 'Milano'
  AND  Dipart = D.Nome -- corrispondenza fra tabelle
\end{lstlisting}


% ---------------------------------------------------------------------------- %
% ---------------------------------------------------------------------------- %

\newpage
\subsection*{Interrogazioni con raggruppamento}

La sintassi SQL non ammette che nella stessa clausola \sql{SELECT} compaiano funzioni aggregate espressioni al livello di riga, a meno che non si faccia uso della \sql{GROUP BY}.

\setcounter{enumi}{26}
	\item Estrarre la somma degli stipendi {\btHL di tutti gli impiegati \textbf{dello stesso dipartimento}}
	\begin{lstlisting}
SELECT    Dipart, sum(Stipendio)
FROM      Impiegato
^GROUP BY Dipart^
\end{lstlisting}

Una applicazione può aver bisogno di considerare solo i sottoinsiemi che soddisfano certe condizioni, questo ci porta alla prossima interrogazione.


\setcounter{enumi}{30}
	\item Estrarre i dipartimenti che spendono più di 10 in Stipendi
	\begin{lstlisting}
SELECT   Dipart, sum(Stipendio)
FROM     Impiegato
GROUP BY Dipart
HAVING   sum(Stipendio) > 100
\end{lstlisting}

La clausola dettata dall'HAVING è \emph{indipendente} da quella del SELECT, quindi potrei restituire anche un'altra proprietà anche se non avrebbe molto senso.



	\item Estrarre i dipartimenti per cui la media degli stipendi degli impiegati che lavorano nell'ufficio 20 è superiore a 25.
	\begin{lstlisting}
SELECT   Dipart
FROM     Impiegato
WHERE    Ufficio = '20'
GROUP BY Dipart
HAVING   avg(Stipendio) > 25
\end{lstlisting}


% ---------------------------------------------------------------------------- %

\subsection*{Interrogazioni di tipo insiemistico}

\begin{itemize}
	\item \texttt{union}
	\item \texttt{intersect}
	\item \texttt{except} (chiamato anche \texttt{minus})
\end{itemize}

Tutte le interrogazioni scritte con questi due costrutti possono essere riscritte tramite interrogazioni nidificate.

	\item Estrarre i nomi e i cognomi dagli impiegati.
	\begin{lstlisting}
SELECT   Nome
FROM     Impiegato
@UNION@
SELECT   Cognome
FROM     Impiegato
\end{lstlisting}

% \begin{tikzpicture}[fill=green!30]
% 	\begin{scope}
% 		\fill \firstcircle \secondcircle;
% 	\end{scope}
% \draw \firstcircle node [text=black, left] {$Nome$};
% \draw \secondcircle node [text=black, right] {$Cognome$};
% \end{tikzpicture}

\begin{table}[H]
	\centering
	\begin{tabular}{@{} l @{}}
		\toprule
			Nome \\
		\midrule
			Mario \\
			Carlo \\
			Giovanni \\
			Franco \\
			Lorenzo \\
			Paola \\
			Marco \\
		\midrule
			Rossi \\
			Bianchi \\
			Verdi \\
			Neri \\
			Gialli \\
			Rosato \\
		\bottomrule
	\end{tabular}
\end{table}


	\item Estrarre i nomi e i cognomi di tutti gli impiegati, eccetto quelli appartenenti al dipartimento Amministrazione, {\btHL mantenendo i duplicati}.
	\begin{lstlisting}
SELECT Nome
FROM   Impiegato
UNION ^ALL^
SELECT Cognome
FROM   Impiegato
WHERE  Dipart <> 'Amministrazione'
\end{lstlisting}


	\item Estrarre i cognomi di impiegati che sono anche nomi.
	\begin{lstlisting}
SELECT Nome
FROM   Impiegato
@INTERSECT@
SELECT Cognome
FROM   Impiegato
\end{lstlisting}

% \begin{tikzpicture}[fill = green!30]
% 	\begin{scope}[even odd rule] % first circle without the second
% 		\clip \secondcircle (-4,-4) rectangle (4,4);
% 	\fill \firstcircle;
% 	\end{scope}
% 	\draw \firstcircle node [text=black, left] {$Nome$};
% 	\draw \secondcircle node [text=black, right] {$Cognome$};
% \end{tikzpicture}

\begin{table}[H]
	\centering
	\begin{tabular}{@{} l @{}}
		\toprule
			Nome \\
		\midrule
		 	Franco \\
		\bottomrule
	\end{tabular}
\end{table}


	\item Estrarre i nomi degli impiegati che non sono cognomi di qualche impiegato.
	\begin{lstlisting}
SELECT Nome
FROM   Impiegato
EXCEPT (minus)
SELECT Cognome
FROM   Impiegato
\end{lstlisting}

\begin{table}[H]
	\centering
	\begin{tabular}{@{} l @{}}
		\toprule
			Nome \\
		\midrule
			Mario \\
			Carlo \\
			Giovanni \\
			Lorenzo \\
			Paola \\
		 	Marco \\
		\bottomrule
	\end{tabular}
\end{table}


% ---------------------------------------------------------------------------- %

\subsection*{Interrogazioni nidificate}

	\item Estrarre gli impiegati che lavorano in dipartimenti situati a Firenze.
	\begin{lstlisting}
SELECT *
FROM   Impiegato
WHERE  Dipart ^= any^ (SELECT Nome
                     FROM   Dipartimento
                     WHERE  Citta = 'Firenze')
\end{lstlisting}

L'interrogazione seleziona le righe di \texttt{Impiegato} per cui il valore dell'attributo \texttt{Dipart} è uguale ad almeno uno dei valori dell'attributo Nome delle righe di \texttt{Dipartimento}.


	\item Consideriamo un'interrogazione che permetta di trovare li impiegati che hanno lo stesso nome di un impiegato del Dipartimento Produzione \dots
	\begin{lstlisting}
SELECT I1.Nome
FROM   Impiegato I1, Impiegato I2
WHERE  I1.Nome = I2.Nome
  AND  I2.Dipart = 'Produzione'
\end{lstlisting}


	\item \dots con operazioni insiemistiche
	\begin{lstlisting}
SELECT Nome
FROM   Impiegato
WHERE  Nome = any (SELECT Nome
                   FROM   Impiegato
                   WHERE  Dipart = 'Produzione')
\end{lstlisting}


	\item Estrarre i nomi dei dipartimenti in cui non lavorano persone di cognome "Rossi" \dots
	\begin{lstlisting}
SELECT Nome
FROM   Dipartimento
EXCEPT
SELECT Dipart
FROM   Impiegato
WHERE  Cognome = 'Rossi'
\end{lstlisting}


	\item \dots con operazioni insiemistiche
	\begin{lstlisting}
SELECT Nome
FROM   Dipartimento
WHERE  Nome <> all (SELECT Dipart
                    FROM   Impiegato
                    WHERE  Cognome = 'Rossi')
\end{lstlisting}

Simboli di appartenenza/non appartenenza rispetto ad un insieme:

\begin{itemize}
	\item \textbf{= any \(\Leftrightarrow\) in}
	\item \textbf{<> all \(\Leftrightarrow\) not in}
\end{itemize}

Interrogazioni che fanno uso degli operatori max e min possono essere rappresentate senza l'uso degli operatori stessi.


	\item Estrarre il dipartimento dell'impiegato che guadagna lo stipendio massimo usando l'operatore \texttt{max} \dots
	\begin{lstlisting}
SELECT Dipart
FROM   Impiegato
WHERE  Stipendio = any (SELECT max(Stipendio)
                        FROM   Impiegato)
\end{lstlisting}


	\item \dots usando solo un'interrogazione nidificata
	\begin{lstlisting}
SELECT Dipart
FROM   Impiegato
WHERE  Stipendio >= all (SELECT Stipendio
                         FROM   Impiegato)
\end{lstlisting}


	\item Estrarre le persone che \textbf{hanno} degli omonimi (stesso nome e cognome ma CF diverso) \dots
	\begin{lstlisting}
SELECT *
FROM   Persona P
WHERE  EXISTS (SELECT *
               FROM   Persona P1
			   WHERE  P.Nome = P1.Nome
			     AND  P.Cognome = P1.Nome
				 AND  P.CF <> P1.CF)
\end{lstlisting}

Per ogni chiamata nidificata devi controllare il valore della tabella superiore.


	\item \dots senza query nidificata
	\begin{lstlisting}
SELECT P.*
FROM   Persona P, Persona P1
WHERE  P1.Nome = P.Nome
  AND  P1.Cognome = P.Cognome
  ANd  P1.CF <> P.CF
\end{lstlisting}


	\item Estrarre le persone che \textbf{non hanno} degli omonimi.
	\begin{lstlisting}
SELECT *
FROM   Persona P
WHERE  NOT EXISTS (SELECT *
                   FROM   Persona P1
                   WHERE  P.Nome = P1.Nome
                   AND  P.Cognome = P1.Nome
                   AND  P.CF <> P1.CF)
\end{lstlisting}


\end{enumerate}

% ---------------------------------------------------------------------------- %

	\newpage
	\input{./assets/other/outro}

% ---------------------------------------------------------------------------- %
% ---------------------------------------------------------------------------- %

\end{document}

% ---------------------------------------------------------------------------- %
% ---------------------------------------------------------------------------- %
% ---------------------------------------------------------------------------- %
